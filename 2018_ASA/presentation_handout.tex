
 % Outline for 10 minute conference presentation ( from http://www.raewynconnell.net/search/label/conferences?&max-results=10)
 % (a) Problem, 2 minutes: Explain very crisply the question you are wrestling with.
 % (b) Main finding, 1 minute 30 seconds. In a conference presentation, this can legitimately come before method and data.
 % (c) Reasoning, 30 seconds + 4 minutes 30 seconds. Say ‘How did I come to this conclusion?’ By doing such-and-such (Method, very very very short); and here is what I found (Materials, in a bit more detail).  Give a slice of your raw material, which really helps the audience to understand. It will be a thin slice, but that’s OK, in a conference you are really offering a sample of your work.
 % (d) Relevance, 2 minutes. Tell the significance of what you have done. If it does suggest something about the literature, say so here at the end: ‘Finally, as you will realize, this finding overthrows both Keynes’ model of effective demand and Einstein’s general theory of relativity’. Whoops, we are 30 seconds over the time limit, will have to trim something.


 %----------------------------------------------------------------------------------------
 %	PACKAGES AND OTHER DOCUMENT CONFIGURATIONS
 %---------------------------------------------------------------------------------------

 \PassOptionsToPackage{unicode=true}{hyperref} % options for packages loaded elsewhere
 \PassOptionsToPackage{hyphens}{url}
 %
 \documentclass[]{article}
 \usepackage{lmodern}
 \usepackage{amssymb,amsmath}
 \usepackage{ifxetex,ifluatex}

  % packages for figures
 \usepackage{array}  % tables
 \usepackage{wrapfig}  % wrap text around narrow tables or figures
 \usepackage{graphicx}  % for inserting graphics from file
 \usepackage{blindtext}
 \usepackage{asymptote}
 \usepackage{subcaption}
 \usepackage[section]{placeins} %keeps floats in place, using command \FloatBarrier
 %\usepackage{endfloat} % forces all floats to appear at end of document
 %\usepackage{flafter} %force floats to appear after they are defined

 % for figures in multiple parts
 \usepackage{caption}
 %\DeclareCaptionLabelFormat{cont}{#1~#2\alph{ContinuedFloat}}
 %\captionsetup[ContinuedFloat]{labelformat=cont}

 \ifnum 0\ifxetex 1\fi\ifluatex 1\fi=0 % if pdftex
   \usepackage[T1]{fontenc}
   \usepackage[utf8]{inputenc}
   \usepackage{textcomp} % provides euro and other symbols
 \else % if luatex or xelatex
   \usepackage{unicode-math}
   \defaultfontfeatures{Ligatures=TeX,Scale=MatchLowercase}
 \fi
 % use upquote if available, for straight quotes in verbatim environments
 \IfFileExists{upquote.sty}{\usepackage{upquote}}{}
 % use microtype if available
 \IfFileExists{microtype.sty}{%
 \usepackage[]{microtype}
 \UseMicrotypeSet[protrusion]{basicmath} % disable protrusion for tt fonts
 }{}
 \IfFileExists{parskip.sty}{%
 \usepackage{parskip}
 }{% else
 \setlength{\parindent}{0pt}
 \setlength{\parskip}{6pt plus 2pt minus 1pt}
 }
 \usepackage{hyperref}
 \hypersetup{
             pdftitle={Information Inequality:},
             pdfborder={0 0 0},
             breaklinks=true}
 \urlstyle{same}  % don't use monospace font for urls
 \usepackage{graphicx,grffile}
 \makeatletter
 \def\maxwidth{\ifdim\Gin@nat@width>\linewidth\linewidth\else\Gin@nat@width\fi}
 \def\maxheight{\ifdim\Gin@nat@height>\textheight\textheight\else\Gin@nat@height\fi}
 \makeatother
 % Scale images if necessary, so that they will not overflow the page
 % margins by default, and it is still possible to overwrite the defaults
 % using explicit options in \includegraphics[width, height, ...]{}
 \setkeys{Gin}{width=\maxwidth,height=\maxheight,keepaspectratio}
 \setlength{\emergencystretch}{3em}  % prevent overfull lines
 \providecommand{\tightlist}{%
   \setlength{\itemsep}{0pt}\setlength{\parskip}{0pt}}
 \setcounter{secnumdepth}{0}
 % Redefines (sub)paragraphs to behave more like sections
 \ifx\paragraph\undefined\else
 \let\oldparagraph\paragraph
 \renewcommand{\paragraph}[1]{\oldparagraph{#1}\mbox{}}
 \fi
 \ifx\subparagraph\undefined\else
 \let\oldsubparagraph\subparagraph
 \renewcommand{\subparagraph}[1]{\oldsubparagraph{#1}\mbox{}}
 \fi


 \pagenumbering{gobble}  % no page numbers
 \usepackage{geometry}
 \geometry{
   hmargin = 1in,
   vmargin = .6in,
   includefoot
 }

 \usepackage{fancyhdr}
 \pagestyle{fancy}
 \fancyhf{}  % Clear header/footer
 \renewcommand{\headrulewidth}{0pt} % could be replaced with 0pt for No header rule


 \fancyfoot[C]{
   \footnotesize
   kingmo@stanford.edu | www.mollymking.com}
   %\unprbottom{1cm}

 %----------------------------------------------------------------------------------------
 %	DOCUMENT CONTENT
 %----------------------------------------------------------------------------------------

 % set default figure placement to htbp
 \makeatletter
 \def\fps@figure{htbp}
 \makeatother

 \begin{document}

 \title{\vspace{-1.0cm}Information Inequality: the Gender of Knowledge}
 \date{}
 \author{Molly M. King  |  PhD Candidate  |  Stanford University}

 \maketitle
\thispagestyle{fancy}

\emph{\textbf{Why care about knowledge inequality?}}

\begin{enumerate}
  \item{Differences in knowledge itself are, by definition, a dimension of `inequality';}
  \item{Knowledge is a potential cause of later inequality in access to resources and outcomes; and}
  \item{Differences in the amount of knowledge people have are influenced by unequal social positions in our society.}
\end{enumerate}


  % Acquisition and use of knowledge
    \begin{figure}[htb]
      \begin{center}
      \includegraphics[width=0.7\linewidth]{"Theoretical Model Drawing"}
      \end{center}
    \end{figure}


\section{Methods}\label{methods}

\emph{\textbf{How does the status gap in knowledge vary by domain?}}

\vspace{7mm}

 % Model
 \begin{table}[ht]
 \centering
     \begin{tabular}{c}  % creates table with text left-centered
       \hline   % adds horizontal lines to the top of the table
       Factors          \\
       \hline   % adds horizontal lines to the top of the table
       \textbf{Gender}           \\
       Income           \\
       Race / Ethnicity \\
       Age + $age^2$    \\
       (Education)        \\
       \hline  % adds horizontal line to the bottom edges
     \end{tabular}
     \begin{tabular}{l}  % creates table with text left-centered
       ---->
     \end{tabular}
     \begin{tabular}{c}  % creates table with text left-centered
        \hline   % adds horizontal lines to the top of the table
        Outcome        \\
        \hline   % adds horizontal lines to the top of the table
        Probability      \\
        of     \\
        getting \\
        question     \\
        correct          \\
        \hline  % adds horizontal line to the bottom edges
      \end{tabular}
 \end{table}

\vspace{7mm}

 \emph{\textbf{Surveys/Data:}} \footnotesize{General Social Survey (5 years); Pew Research Center (21); Kaiser Family Foundation; Health Information National Trends Survey (8); Integrated Health Interview Series; Annenberg National Health Communication Survey; USC's Understanding America Study (3); Rand American Life Panel (2); National Financial Capability Studies (3); 21st Century Americanism survey; Global Views American Public Opinion and Foreign Policy; Outlook on Life Survey; State of the First Amendment surveys; Chicago Survey of American Public Opinion and U.S. Foreign Policy}

\newpage
\section{Results}\label{Results}

% Cross-domain gender log odds - regression with education
\emph{\textbf{How likely are the wealthy to get each answer correct , controlling for all else?}}
\begin{figure}[ht]
    \begin{center}
      \includegraphics[height=3.20in]{"/Users/mollymking/Documents/SocResearch/Dissertation/results/figures/08_vis/ki08_vis06_graphCIs_C_edu_all-income"}
      \caption{Income correlates with mean knowledge advantage in 15/16 domains}
    \end{center}
\end{figure}

\vspace{5mm}

\emph{\textbf{What proportion of questions are the wealthy more likely to answer correctly?}}
\begin{figure}[ht]
    \begin{center}
      \includegraphics[height=3.20in]{"/Users/mollymking/Documents/SocResearch/Dissertation/results/figures/08_vis/ki08_vis03_propHoriz02_multComp_C_edu_income"}
      \caption{Those with higher incomes answer a greater proportion of questions correctly}
    \end{center}
\end{figure}

\emph{\textbf{What proportion of questions are Blacks and whites more likely to answer correctly?}}
\begin{figure}[ht]
    \begin{center}
      \includegraphics[height=3.20in]{"/Users/mollymking/Documents/SocResearch/Dissertation/results/figures/08_vis/ki08_vis03_propHoriz02_multComp_C_edu_rblack"}
      \caption{Proportion of questions answered by non-Hispanic Whites vs Blacks}
    \end{center}
\end{figure}

\end{document}
