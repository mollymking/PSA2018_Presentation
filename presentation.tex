\documentclass[pdf]{beamer}
\usetheme{Boadilla}
\mode<presentation>{}

 % % preamble
\title{Information Inequality}
\subtitle{the Class, Gender, and Race of Knowledge Domains}
\author{Molly King}
\institute{Stanford University}
\date{March 31, 2018}

\begin{document}

 % Title Slide
\begin{frame}
  \titlepage
\end{frame}


 % Table Of Contents Slide
\begin{frame}
\frametitle{Outline}
  \tableofcontents
\end{frame}

\section{Introduction}

% Acquisition and use of knowledge
\subsection{Model of knowledge}
\begin{frame}
\frametitle{Acquisition and use of knowledge}
 \begin{figure}[ht]
   \begin{center}
   \includegraphics[width=\linewidth]{"/Users/mollymking/Documents/SocResearch/Dissertation/results/figures/KnowledgeAcquisitionUse"}
   \end{center}
 \end{figure}
\end{frame}


%Motivation - 3 reasons
\subsection{Motivation}
\begin{frame}
\frametitle{Why care about information inequality?}
  \begin{itemize}
\item
  Differences in the amount of information people have are influenced by
  unequal social positions in our society;
\pause
\item
  Information is a potential cause of later inequality in outcomes and
  access to resources; and
\pause
\item
  Differences in information capacity itself are, by definition, a
  dimension of `inequality.'
  \end{itemize}
\end{frame}


% Research Questions
\subsection{Research Question}
\begin{frame}
\frametitle{Research Question}
  \emph{How does the status gap in knowledge vary by domain?}
     % The field of sociology has long studied the production of knowledge in science inequalities in knowledge careers; and information diffusion and its consequences. Many studies have evaluated information seeking behaviors and needs. But the tendency has either been to study knowledge in one specific domain (e.g., health) or to reduce knowledge across all domains to a single test score -- and hence we know shockingly little about the everyday knowledge stock of Americans.
\end{frame}



% Methods
\section{Methods}

 % Data
\subsection{Data}
\begin{frame}
\frametitle{Data}
  \begin{columns}
      \begin{tabular}{l}  % creates table with text left-centered, vertical lines right and left
        \hline   % adds horizontal lines to the top of the table
        General Social Survey           \\
        Pew Research Center (21)           \\
        Kaiser Family Foundation        \\
        Health Information National Trends Survey (8) \\
        Integrated Health Interview Series \\
        Annenberg National Health Communication Survey \\
        USC's Understanding America Study (3) \\
        Rand American Life Panel (2)  \\
        National Financial Capability Studies (3) \\
        21st Century Americanism survey \\
        Global Views American Public Opinion and Foreign Policy \\
        Outlook on Life Survey \\
        State of the First Amendment surveys \\
        Chicago Council Survey of American Public Opinion and U.S. Foreign Policy \\
        \hline  % adds horizontal line to the bottom edges
        \end{tabular}
   \end{columns}
\end{frame}

\subsection{Domains}
\begin{frame}
\frametitle{Domains}
  \begin{columns}
    \column{0.5\textwidth}
      \begin{tabular} {l}  % creates table with text left-centered, vertical lines right and left
        \hline   % adds horizontal lines to the top of the table
        history			                   \\
        natural world                  \\
        physical science               \\
        biological science             \\
        technology                     \\
        math                           \\
        culture                        \\
        geography                      \\
        domestic politics              \\
        foreign politics               \\
        economics                      \\
        finance                        \\
        health                         \\
        religion                       \\
        pop culture                    \\
        war                            \\
        \hline  % adds horizontal line to the bottom edges
        \end{tabular}
  \end{columns}
\end{frame}


\section{Results}
\subsection{Means across domains}
\subsection{Proportions correct by group}

 % Cross-domain gender log odds
\begin{frame}
\frametitle{No mean gender difference in 5/16 domains}
  \begin{figure}[ht]
    \begin{center}
      \includegraphics[height=3in]{"/Users/mollymking/Documents/SocResearch/Dissertation/results/figures/08_vis/ki08_vis06_graphCIs_all-gender"}
    \end{center}
  \end{figure}
\end{frame}


% Proportion of questions above and below 0 log odds by gender
\begin{frame}
\frametitle{Men answer greater proportion of questions correctly in 65\% of domains}
  \begin{figure}[ht]
    \begin{center}
      \includegraphics[height=3in]{"/Users/mollymking/Documents/SocResearch/Dissertation/results/figures/08_vis/ki08_vis03_propHoriz02_multComp_gender"}
    \end{center}
  \end{figure}
\end{frame}


% Log odds for correct answer by class and knowledge domain
\begin{frame}
\frametitle{Men answer greater proportion of questions correctly in 65\% of domains}
  \begin{figure}[ht]
    \begin{center}
    \includegraphics[height=3in]{"/Users/mollymking/Documents/SocResearch/Dissertation/results/figures/08_vis/ki08_vis06_graphCIs_all-income"}
    \end{center}
  \end{figure}
\end{frame}


% Proportion of questions above and below 0 log odds by income
\begin{frame}
\frametitle{}
  \begin{figure}[ht]
    \begin{center}
    \includegraphics[height=3in]{"/Users/mollymking/Documents/SocResearch/Dissertation/results/figures/08_vis/ki08_vis03_propHoriz02_multComp_income"}
    \end{center}
  \end{figure}
\end{frame}


% Log odds for correct answer by Race and knowledge domain
\begin{frame}
\frametitle{}
  \begin{figure}[ht]
    \begin{center}
    \includegraphics[height=3in]{"/Users/mollymking/Documents/SocResearch/Dissertation/results/figures/08_vis/ki08_vis06_graphCIs_all-race_byGroup"}
    \end{center}
  \end{figure}
\end{frame}


\end{document}
