\documentclass[pdf]{beamer}
\usepackage{pgfpages}
\setbeameroption{show notes on second screen = right}  % tells to show notes on right hand side screen
\usetheme{Boadilla}
\mode<presentation>{}
% To give a presentation with the Skim reader (http://skim-app.sourceforge.net) on OSX so
% that you see the notes on your laptop and the slides on the projector, do the following:
%
% 1. Generate just the presentation (hide notes) and save to slides.
%\setbeameroption{hide notes} % Only slides
% 2. Generate only the notes (show only nodes) and save to notes.pdf
%\setbeameroption{show only notes} % Only notes

% 3. With Skim open both slides.pdf and notes.pdf
% 4. Click on slides.pdf to bring it to front.
% 5. In Skim, under "View -> Presentation Option -> Synhcronized Noted Document"
%    select notes.pdf.
% 6. Now as you move around in slides.pdf the notes.pdf file will follow you.
% 7. Arrange windows so that notes.pdf is in full screen mode on your laptop
%    and slides.pdf is in presentation mode on the projector.

% Give a slight yellow tint to the notes page
\setbeamertemplate{note page}{\pagecolor{yellow!5}\insertnote}\usepackage{palatino}

 % % preamble
\title{Information Inequality}
\subtitle{the Class, Gender, and Race of Knowledge Domains}
\author{Molly King}
\institute{Stanford University}
\date{March 31, 2018}

\begin{document}

 % Title Slide
\begin{frame}
  \titlepage
\end{frame}


 % Table Of Contents Slide
\begin{frame}
\frametitle{Outline}
  \tableofcontents
\end{frame}

\section{Introduction}

%Motivation - 3 reasons
\subsection{Motivation}
\begin{frame}
\frametitle{Why care about information inequality?}
  \begin{itemize}
  \note[item]{Most of us are in this room because we try to understand the origins or cures of inequality.}

  \item
    Differences in information capacity itself are, by definition, a
    dimension of `inequality';
      \note[item]{Were also all here because we like learning for the sake of knowledge, so I probably don't have to argue too much for this idea of the current value of knowledge.}
  \pause
  \item
    Differences in the amount of information people have are influenced by
    unequal social positions in our society; and
      \note[item]{I argue that a concept I am calling information inequality - or knowledge inequality - is important as both an outcome and cause of social inequality.}
  \pause
  \item
    Information is a potential cause of later inequality in outcomes and
    access to resources.

  \note[item]{So, while I argue that knowledge inequality is important from both ends of the causal argument, in this research I focus on the idea that social status causes knowledge inequality.}
  \end{itemize}
\end{frame}


% Research Questions
\subsection{Research Question}
\begin{frame}
\frametitle{Research Question}
  \emph{How does the status gap in knowledge vary by domain?}
     % The field of sociology has long studied the production of knowledge in science inequalities in knowledge careers; and information diffusion and its consequences. Many studies have evaluated information seeking behaviors and needs. But the tendency has either been to study knowledge in one specific domain (e.g., health) or to reduce knowledge across all domains to a single test score -- and hence we know shockingly little about the everyday knowledge stock of Americans.
\end{frame}



% Methods
\section{Methods}

 % Data
\subsection{Data}
\begin{frame}
\frametitle{Data}
  \begin{columns}
      \begin{tabular}{l}  % creates table with text left-centered, vertical lines right and left
        \hline   % adds horizontal lines to the top of the table
        General Social Survey           \\
        Pew Research Center (21)           \\
        Kaiser Family Foundation        \\
        Health Information National Trends Survey (8) \\
        Integrated Health Interview Series \\
        Annenberg National Health Communication Survey \\
        USC's Understanding America Study (3) \\
        Rand American Life Panel (2)  \\
        National Financial Capability Studies (3) \\
        21st Century Americanism survey \\
        Global Views American Public Opinion and Foreign Policy \\
        Outlook on Life Survey \\
        State of the First Amendment surveys \\
        Chicago Council Survey of American Public Opinion and U.S. Foreign Policy \\
        \hline  % adds horizontal line to the bottom edges
        \end{tabular}
   \end{columns}
\end{frame}

\subsection{Domains}
\begin{frame}
\frametitle{Domains}
  \begin{columns}
    \column{0.5\textwidth}
      \begin{tabular} {l}  % creates table with text left-centered, vertical lines right and left
        \hline   % adds horizontal lines to the top of the table
        history			                   \\
        natural world                  \\
        physical science               \\
        biological science             \\
        technology                     \\
        math                           \\
        culture                        \\
        geography                      \\
        domestic politics              \\
        foreign politics               \\
        economics                      \\
        finance                        \\
        health                         \\
        religion                       \\
        pop culture                    \\
        war                            \\
        \hline  % adds horizontal line to the bottom edges
        \end{tabular}
  \end{columns}
\end{frame}


\section{Results}
\subsection{Means across domains}
\subsection{Proportions correct by group}

% Log odds for correct answer by class and knowledge domain
\begin{frame}
\frametitle{}
  \begin{figure}[ht]
    \begin{center}
    \includegraphics[height=3in]{"/Users/mollymking/Documents/SocResearch/Dissertation/results/figures/08_vis/ki08_vis06_graphCIs_all-income"}
    \end{center}
  \end{figure}
\end{frame}


% Proportion of questions above and below 0 log odds by income
\begin{frame}
\frametitle{}
  \begin{figure}[ht]
    \begin{center}
    \includegraphics[height=3in]{"/Users/mollymking/Documents/SocResearch/Dissertation/results/figures/08_vis/ki08_vis03_propHoriz02_multComp_income"}
    \end{center}
  \end{figure}
\end{frame}


 % Cross-domain gender log odds
\begin{frame}
\frametitle{No mean gender difference in 5/16 domains}
  \begin{figure}[ht]
    \begin{center}
      \includegraphics[height=3in]{"/Users/mollymking/Documents/SocResearch/Dissertation/results/figures/08_vis/ki08_vis06_graphCIs_all-gender"}
    \end{center}
  \end{figure}
\end{frame}


% Proportion of questions above and below 0 log odds by gender
\begin{frame}
\frametitle{Men answer greater proportion of questions correctly in 65\% of domains}
  \begin{figure}[ht]
    \begin{center}
      \includegraphics[height=3in]{"/Users/mollymking/Documents/SocResearch/Dissertation/results/figures/08_vis/ki08_vis03_propHoriz02_multComp_gender"}
    \end{center}
  \end{figure}
\end{frame}


% Log odds for correct answer by Race and knowledge domain
\begin{frame}
\frametitle{}
  \begin{figure}[ht]
    \begin{center}
    \includegraphics[height=3in]{"/Users/mollymking/Documents/SocResearch/Dissertation/results/figures/08_vis/ki08_vis06_graphCIs_all-race_byGroup"}
    \end{center}
  \end{figure}
\end{frame}



\section{Implications}

% Acquisition and use of knowledge
\subsection{Model of knowledge}
\begin{frame}
\frametitle{Acquisition and use of knowledge}
  \begin{figure}[ht]
    \begin{center}
      \includegraphics[width=\linewidth]{"/Users/mollymking/Documents/SocResearch/Dissertation/results/figures/KnowledgeAcquisitionUse"}
    \end{center}
  \end{figure

  \note[item]{Findings are consistent with the model of using knowledge to access resources.}
  \note[item]{Understanding the broad demographic patterns can help us move toward better understanding of the mechanisms behind them.}
  \note[item]{}
\end{frame}




\end{document}
